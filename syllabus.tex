%Template found at https://www.overleaf.com/latex/templates/syllabus-template-course-info/gbqbpcdgvxjs

\documentclass[11pt, a4paper]{article}
\usepackage[inner=1.5cm,outer=1.5cm,top=2.5cm,bottom=2.5cm]{geometry}
\pagestyle{empty}
\usepackage{graphicx}
\usepackage{fancyhdr, lastpage, bbding, pmboxdraw}
\usepackage[usenames,dvipsnames]{color}
\definecolor{darkblue}{rgb}{0,0,.6}
\definecolor{darkred}{rgb}{.7,0,0}
\definecolor{darkgreen}{rgb}{0,.6,0}
\definecolor{red}{rgb}{.98,0,0}
\usepackage[colorlinks,pagebackref,pdfusetitle,urlcolor=darkblue,citecolor=darkblue,linkcolor=darkred,bookmarksnumbered,plainpages=false]{hyperref}
\renewcommand{\thefootnote}{\fnsymbol{footnote}}

\pagestyle{fancyplain}
\fancyhf{}
\lhead{ \fancyplain{}{Course Name} }
%\chead{ \fancyplain{}{} }
\rhead{ \fancyplain{}{\today} }
%\rfoot{\fancyplain{}{page \thepage\ of \pageref{LastPage}}}
%\fancyfoot[RO, LE] {page \thepage\ of \pageref{LastPage} }
\thispagestyle{plain}

%%%%%%%%%%%% LISTING %%%
\usepackage{listings}
\usepackage{caption}
\DeclareCaptionFont{white}{\color{white}}
\DeclareCaptionFormat{listing}{\colorbox{gray}{\parbox{\textwidth}{#1#2#3}}}
\captionsetup[lstlisting]{format=listing,labelfont=white,textfont=white}
\usepackage{verbatim} % used to display code
\usepackage{fancyvrb}
\usepackage{acronym}
\usepackage{amsthm}
\VerbatimFootnotes % Required, otherwise verbatim does not work in footnotes!



\definecolor{OliveGreen}{cmyk}{0.64,0,0.95,0.40}
\definecolor{CadetBlue}{cmyk}{0.62,0.57,0.23,0}
\definecolor{lightlightgray}{gray}{0.93}



\lstset{
%language=bash,                          % Code langugage
basicstyle=\ttfamily,                   % Code font, Examples: \footnotesize, \ttfamily
keywordstyle=\color{OliveGreen},        % Keywords font ('*' = uppercase)
commentstyle=\color{gray},              % Comments font
numbers=left,                           % Line nums position
numberstyle=\tiny,                      % Line-numbers fonts
stepnumber=1,                           % Step between two line-numbers
numbersep=5pt,                          % How far are line-numbers from code
backgroundcolor=\color{lightlightgray}, % Choose background color
frame=none,                             % A frame around the code
tabsize=2,                              % Default tab size
captionpos=t,                           % Caption-position = bottom
breaklines=true,                        % Automatic line breaking?
breakatwhitespace=false,                % Automatic breaks only at whitespace?
showspaces=false,                       % Dont make spaces visible
showtabs=false,                         % Dont make tabls visible
columns=flexible,                       % Column format
morekeywords={__global__, __device__},  % CUDA specific keywords
}

%%%%%%%%%%%%%%%%%%%%%%%%%%%%%%%%%%%%
\begin{document}
\begin{center}
{\Large \textsc{Probability and Statistics}}
\end{center}
\begin{center}
Summer 2019
\end{center}

\begin{center}
\rule{6in}{0.4pt}
\begin{minipage}[t]{.75\textwidth}
\begin{tabular}{llcccll}
\textbf{Instructor:} & Jonathan Adams & & &  & \textbf{Time:} & MTWRF 12:00 -- 12:55 \\
\textbf{Email:} &  jd.adams16@gmail.com & & & & \textbf{Place:} & Math 11
\end{tabular}
\end{minipage}
\rule{6in}{0.4pt}
\end{center}
\vspace{.5cm}
\setlength{\unitlength}{1in}
\renewcommand{\arraystretch}{2}

\noindent\textbf{Course Page:} https://github.com/PieceMaker/texprep-probability-and-statistics

\vskip.15in
\noindent\textbf{Office Hours:} Office hours will not be available for this course. If you have questions, email me or ask Ms. Stamm.

\vskip.15in
\noindent\textbf{Main Reference:}
The book that will be used for this course is not required. The information is provided here solely for future reference due to personal interest.
\begin{itemize}
\item Dennis D. Wackerly, William Mendenhall III, and Richard L. Scheaffer, {\textit{Mathematical Statistics with Applications}}, Brooks/Cole, 7th ed., 2008.
\end{itemize}

\vskip.15in
\noindent\textbf{Objectives:}  The goal of this course is to give a simple introduction to the field of probability and statistics. This will be accomplished by a combination of hands-on experiments with data analysis as well as in-class lecture. Some of the topics include:

\begin{itemize}
\item Laws of Probability
\item Counting/Combinatorics
\item Introduction to Discrete Distributions
\item Design and Analysis of Experiments
\end{itemize}

\vspace*{.15in}

\vspace*{.15in}
\noindent\textbf{Grading Policy:} Homework (30\%), Projects (30\%),  Midterm (20\%), Final (20\%).

\vskip.15in
\noindent\textbf{Important Dates:}
\begin{center} \begin{minipage}{3.8in}
\begin{flushleft}
Midterm Questions \dotfill June 14 \\
Midterm      \dotfill June 19  \\
Final Questions \dotfill June 28 \\
Final       \dotfill July 3  \\
\end{flushleft}
\end{minipage}
\end{center}

\vskip.15in
\noindent\textbf{Course Policy:}  
\begin{itemize}
\item Regular attendance is essential and expected.
\item Late work will only be accepted when directed to by the TexPREP office.
\item No makeup exams will be given unless directed to by the TexPREP office.
\item You are expected to act in a civil manner toward the teacher, assistant, and fellow students.
\item Cheating will result in a 0\% for the assignment.
\end{itemize}

\end{document} 