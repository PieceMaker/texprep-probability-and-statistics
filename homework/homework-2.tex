% Template from https://www.overleaf.com/latex/templates/chapter-review-notes/npqqbrvfkwqh
\documentclass[11pt]{article}
\usepackage{gensymb}
\usepackage[utf8]{inputenc}	
\usepackage{amsmath,amsthm,amsfonts,amssymb,amscd}
\usepackage{multirow,booktabs}
\usepackage[table]{xcolor}
\usepackage{fullpage}
\usepackage{lastpage}
\usepackage{enumitem}
\usepackage{fancyhdr}
\usepackage{mathrsfs}
\usepackage{wrapfig}
\usepackage{setspace}
\usepackage{calc}
\usepackage{multicol}
\usepackage{cancel}
\usepackage[retainorgcmds]{IEEEtrantools}
\usepackage[margin=3cm]{geometry}
\usepackage{amsmath}
\usepackage{float}
\newlength{\tabcont}
\setlength{\parindent}{0.0in}
\setlength{\parskip}{0.05in}
\usepackage{empheq}
\usepackage{framed}
\usepackage[most]{tcolorbox}
\usepackage{xcolor}
\colorlet{shadecolor}{orange!15}
\parindent 0in
\parskip 12pt
\geometry{margin=1in, headsep=0.25in}
\theoremstyle{definition}
\newtheorem{defn}{Definition}
\newtheorem{reg}{Rule}
\newtheorem{exer}{Exercise}
\newtheorem{note}{Note}
\begin{document}
\title{Title}

\thispagestyle{empty}

\begin{center}
{\LARGE \bf Homework 2} \\
{\large Probability \& Statistics} \\
{\large Due June 11th}
\end{center}

\section*{Problem 1 (4 pts.)}

A box contains three marbles: one red, one green, and one blue. Consider an experiment that consists of taking one marble from the box then replacing it in the box and drawing a second marble from the box.

\begin{enumerate}
	\item What is the sample space?
	\item If each marble in the box is equally likely to be selected, what is the probability of each point in the sample space?
\end{enumerate}

\section*{Problem 2 (6 pts.)}

An oil prospecting firm hits oil or gas on 10\% of its drillings. If the firm drills two wells, the four possible simple events and three of their associated probabilities are as given in the accompanying table. Find the probability that the company will hit oil or gas

\begin{enumerate}
	\item on the first drilling and miss on the second.
	\item on at least one of the two drillings.
\end{enumerate}

\begin{table}[H]
	\centering
	\begin{tabular}{lllc}
		\hline
		Simple & Outcome of & Outcome of & \\
		Event & First Drilling & Second Drilling & Probability \\
		\hline
		$E_1$ & Hit (oil or gas) & Hit (oil or gas) & 0.01 \\
		$E_2$ & Hit & Miss & $x$ \\
		$E_3$ & Miss & Hit & 0.09 \\
		$E_4$ & Miss & Miss & 0.81 \\
		\hline
	\end{tabular}
\end{table}

\end{document}
