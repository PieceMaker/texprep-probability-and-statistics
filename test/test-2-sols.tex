% Template from https://www.overleaf.com/latex/templates/chapter-review-notes/npqqbrvfkwqh
\documentclass[11pt]{article}
\usepackage{gensymb}
\usepackage[utf8]{inputenc}	
\usepackage{amsmath,amsthm,amsfonts,amssymb,amscd}
\usepackage{multirow,booktabs}
\usepackage[table]{xcolor}
\usepackage{fullpage}
\usepackage{lastpage}
\usepackage{enumitem}
\usepackage{fancyhdr}
\usepackage{mathrsfs}
\usepackage{wrapfig}
\usepackage{setspace}
\usepackage{calc}
\usepackage{multicol}
\usepackage{cancel}
\usepackage[retainorgcmds]{IEEEtrantools}
\usepackage[margin=3cm]{geometry}
\usepackage{amsmath}
\newlength{\tabcont}
\setlength{\parindent}{0.0in}
\setlength{\parskip}{0.05in}
\usepackage{empheq}
\usepackage{framed}
\usepackage[most]{tcolorbox}
\usepackage{xcolor}
\colorlet{shadecolor}{orange!15}
\parindent 0in
\parskip 12pt
\geometry{margin=1in, headsep=0.25in}
\theoremstyle{definition}
\newtheorem{defn}{Definition}
\newtheorem{reg}{Rule}
\newtheorem{exer}{Exercise}
\newtheorem{note}{Note}
\begin{document}
\title{Title}

\thispagestyle{empty}

\begin{center}
{\LARGE \bf Test 2} \\
{\large Probability \& Statistics} \\
{\large Show all work. Write all answers on exam. Use back of page when necessary. } \\
\end{center}
\section*{Question 1 (20 pts.)}

Let $Y$ be a random variable with distribution $p(y)$ given in the following table:

\begin{table}[h]
	\centering
	\begin{tabular}{lr}
		\hline
		y & $p(y)$ \\
		\hline
		1 & 0.4 \\
		2 & 0.3 \\
		3 & 0.2 \\
		4 & 0.1 \\
		\hline
	\end{tabular}
\end{table}

\noindent Calculate $E[Y]$, $E[5Y+2]$, $E\left[ Y^2 \right]$, and $V[Y]$. Use any valid method to calculate $V[Y]$.

\section*{Solution}

$$
\begin{aligned}
	E[Y] & = 1 \cdot 0.4 + 2 \cdot 0.3 + 3 \cdot 0.2 + 4 \cdot 0.1 = 0.4 + 0.6 + 0.6 + 0.4 = 2 \\
	E[5Y+2] & = 5E[Y]+2 = 5\cdot2 +2 = 12 \\
	E\left[ Y^2 \right] & = 1 \cdot 0.4 + 4 \cdot 0.3 + 9 \cdot 0.2 + 16 \cdot 0.1 = 0.4 + 1.2 + 1.8 + 1.6 = 5 \\
	V[Y] & = E\left[ Y^2 \right] - \left( E[Y] \right)^2 = 5 - 4 = 1
\end{aligned}
$$

\newpage

\section*{Question 2 (20 pts.)}

You are given

$$
	\begin{aligned}
		E[Y] & = 11 \\
		E[Y^2] & = 169 \\
		E[Y^3] & = 2500
	\end{aligned}
$$

\noindent Calculate the following:

\begin{enumerate}
	\item $E[2Y - 3Y^2 + 5Y^3]$
	\item $V[Y]$
\end{enumerate}

\section*{Solution}

$$
	\begin{aligned}
		E\left[ 2Y - 3Y^2 + 5Y^3 \right] & = 2E[Y] - 3E\left[ Y^2 \right] + 5 E\left[ Y^3 \right] \\
		& = 2 \cdot 11 - 3 \cdot 169 + 5 \cdot 2500 \\
		& = 22 - 507 + 12500 = 12015 \\
		V[Y] & = E\left[ Y^2 \right] - \left( E[Y] \right)^2 \\
		& = 169 - 121 = 48
	\end{aligned}
$$

\newpage

\section*{Question 3 (20 pts.)}

A potential customer for an \$85,000 fire insurance policy possesses a home in an area that, according to experience, may sustain a total (100\%) loss in a given year with probability 0.005 and a 50\% loss with probability 0.05. Assuming these are the only losses the policy can sustain, show that the premium that must be charged on all \$85,000 policies in this area is \$2,550.

\section*{Solution}

Let $C$ be the premium.

\[ \text{Income} = \begin{cases} 
          C, & \text{no loss} \\
			C - 42500, & \text{half loss} \\
			C - 85000, & \text{full loss}
       \end{cases}
\]

The solution is the premium that results in an expected income of 0.

$$
	\begin{aligned}
		E[\text{Income}] & = C \cdot 0.945 + (C - 42500) \cdot 0.05 + (C - 85000) \cdot 0.005 \\
		& = C - 2125 - 425 = C - 2550 = 0 \\
		& \therefore C = 2550
	\end{aligned}
$$

\newpage

\section*{Question 4 (20 pts.)}

Show that the binomial distribution satisfies the second law of probability, i.e. show that the sum of all probabilities of the binomial distribution sums to 1.

\section*{Solution}

$$
	p(y) = {n \choose y} p^y q^{n-y}, \; y = 0,1,2,\ldots,n \; \text{and} \; 0 \leq p \leq 1, \; q = 1-p
$$

$$
	\begin{aligned}
	\sum_{i=0}^n {n \choose i} q^{n-i}p^i & = {n \choose 0} q^n p^0 + {n \choose 1} q^{n-1}p^1 + {n \choose 2} q^{n-2}p^2 + \cdots + {n \choose n} q^0 p^n \\
	& = (q + p)^n = (1 - p + p)^n = 1^n = 1
	\end{aligned}
$$

\newpage

\section*{Question 5 (20 pts.)}

A fire-detection device utilizes three temperature-sensitive cells acting independently of each other in such a manner that any one or more may activate the alarm. Each cell has a probability of 0.8 of activating the alarm when the temperature reaches $100^\circ$ Celsius or more. Let $Y$ equal the number of cells activating the alarm when the temperature reaches $100^\circ$ Celsius.

\begin{enumerate}
	\item What type of experiment is this? Why?
	\item What is the probability that the alarm will function when the temperature reaches $100^\circ$ Celsius?
	\item Calculate $E[Y]$ and $V[Y]$.
\end{enumerate}

\section*{Solution}

\begin{enumerate}
	\item This is a binomial experiment because there are a set number of trials, they are all independent, they have the same success probability, and there are only two outcomes. We are also interested in the total number of successes.
	\item Let $Y$ be the number of cells triggering the alarm. Then the probability of success is $p=0.8$.
	
	$$
		\begin{aligned}
			P(\text{the alarm functions}) & = P(Y \geq 1) = 1 - P(Y=0) \\
			& = 1 - {3 \choose 0} 0.8^0 0.2^3 = 1-0.2^3 \\
			& = 0.992
		\end{aligned}
	$$
	\item
	$$
		\begin{aligned}
			E[Y] & = np = 3 \cdot 0.8 = 2.4 \\
			V[Y] & = npq = 3 \cdot 0.8 \cdot 0.2 = 0.48
		\end{aligned}
	$$
\end{enumerate}

\newpage

\section*{Question 6 (20 pts.)}

A certified public accountant (CPA) has found that 4 of 10 company audits contain substantial errors. If the CPA audits a series of company accounts, let $Y$ equal the audit number of the first company that has substantial errors.

\begin{enumerate}
	\item What type of experiment is this? Why?
	\item What is a success? What is the probability of success?
	\item What is the probability that the first company account containing substantial errors is the third one to be audited?
	\item What is the probability that the first company account to contain substantial errors will occur on or after the third audited account?
\end{enumerate}

\section*{Solution}

\begin{enumerate}
	\item Geometric, because each trial is independent, has the same probability of success, only has two outcomes, and we want to know when the first success occurs.
	\item A success is the company audit having substantial errors. The probability of success is $p = \frac{4}{10} = 0.4$.
	\item
	Let $Y$ be the number of the first audit containing substantial errors.
		$$
			\begin{aligned}
				P(\text{first substantial errors are found on the third audit}) & = P(Y=3) = q^2 p \\
				& = 0.6^2 \cdot 0.4 = 0.144
			\end{aligned}
		$$
	\item
		$$
			\begin{aligned}
				P(\text{first substantial errors are on or after the third audit}) & = P(Y\geq 3) = 1 - P(Y \leq 2) \\
				& = 1 - P(Y = 1) - P(Y = 2) \\
				& = 1 - p - qp \\
				& = 1 - 0.4 - 0.6 \cdot 0.4 = 0.36
			\end{aligned}
		$$
\end{enumerate}

\end{document}
