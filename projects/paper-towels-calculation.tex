% --------------------------------------------------------------
% This is all preamble stuff that you don't have to worry about.
% Head down to where it says "Start here"
% --------------------------------------------------------------
 
\documentclass[12pt]{article}
 
\usepackage[margin=1in]{geometry} 
\usepackage{amsmath,amsthm,amssymb}
 
\newcommand{\N}{\mathbb{N}}
\newcommand{\Z}{\mathbb{Z}}
 
\newenvironment{theorem}[2][Theorem]{\begin{trivlist}
\item[\hskip \labelsep {\bfseries #1}\hskip \labelsep {\bfseries #2.}]}{\end{trivlist}}
\newenvironment{lemma}[2][Lemma]{\begin{trivlist}
\item[\hskip \labelsep {\bfseries #1}\hskip \labelsep {\bfseries #2.}]}{\end{trivlist}}
\newenvironment{exercise}[2][Exercise]{\begin{trivlist}
\item[\hskip \labelsep {\bfseries #1}\hskip \labelsep {\bfseries #2.}]}{\end{trivlist}}
\newenvironment{problem}[2][Problem]{\begin{trivlist}
\item[\hskip \labelsep {\bfseries #1}\hskip \labelsep {\bfseries #2.}]}{\end{trivlist}}
\newenvironment{question}[2][Question]{\begin{trivlist}
\item[\hskip \labelsep {\bfseries #1}\hskip \labelsep {\bfseries #2.}]}{\end{trivlist}}
\newenvironment{corollary}[2][Corollary]{\begin{trivlist}
\item[\hskip \labelsep {\bfseries #1}\hskip \labelsep {\bfseries #2.}]}{\end{trivlist}}

\newenvironment{solution}{\begin{proof}[Solution]}{\end{proof}}
 
\begin{document}
 
\title{Project 1: Paper Towels Part 2}
\date{}

\maketitle

We have visually analyzed the paper towel data collected on the first day of class, but in order to conclude whether paper towels A differ from paper towels B we must perform a numerical test. As discussed in class, this test will involve applying the empirical rule to calculate intervals and comparing these intervals to make the final conclusion.

As a quick recap, the second interval of the empirical rule states that, if the data are bell-shaped, then approximately 95\% of the data can be found in the interval

$$
(\mu - 2\sigma, \mu + 2\sigma)
$$

Since $\mu$ and $\sigma$ are the mean and standard deviation of the \textit{population}, which we do not know, we then use the sample mean ($\bar{y}$) and sample standard deviation ($s$).

$$
(\bar{y} - 2s, \bar{y} + 2s)
$$

Finally, we wish to account for the sample size so we use the following interval for our test:

$$
\left(\bar{y} - 2\frac{s}{\sqrt{n}}, \bar{y} + 2\frac{s}{\sqrt{n}}\right)
$$

In this part of the project, perform the following tasks:

\begin{enumerate}
	\item The data collected by the 6 groups is on the back of this sheet. For each group (using only the data collected by that group), use the ``weights" value to calculate $\bar{y}$ and $s$ for paper towel A and $\bar{y}$ and $s$ for paper towel B. You should have two $\bar{y}$'s and $s$'s per group. Round your answers to 3 decimal places.
	\begin{enumerate}
		\item For paper towel A in group 5, where the ``weights" value is ``NA", ignore this value and calculate $\bar{y}$ and $s$ with the remaining 4 values. Remember this means $n = 4$.
	\end{enumerate}
	\item Assume the data are bell-shaped. For each group and paper towel, use the third interval equation above (the one accounting for sample size) to calculate interval using the $\bar{y}$'s and $s$'s calculated in step 1. Round to 3 decimal places.
	\item For each group, compare the two paper towel intervals calculated in step 2 and conclude whether the results from that group suggest that paper towels A and B differ or do not differ.
\end{enumerate}

\textbf{Record your answers to steps 1, 2, and 3 on a separate sheet of paper. This will be what you turn in.}

\begin{table}[ht]
\centering
\begin{tabular}{cccc}
  \hline
group & trial & towel & weights \\ 
  \hline
  1 &   1 & A &  27 \\ 
    1 &   2 & A &  20 \\ 
    1 &   3 & A &  16 \\ 
    1 &   4 & A &  28 \\ 
    1 &   5 & A &  23 \\ 
    1 &   1 & B &  16 \\ 
    1 &   2 & B &  16 \\ 
    1 &   3 & B &  14 \\ 
    1 &   4 & B &  14 \\ 
    1 &   5 & B &  10 \\ 
    2 &   1 & A &  28 \\ 
    2 &   2 & A &  28 \\ 
    2 &   3 & A &  16 \\ 
    2 &   4 & A &  24 \\ 
    2 &   5 & A &  27 \\ 
    2 &   1 & B &  11 \\ 
    2 &   2 & B &  12 \\ 
    2 &   3 & B &  12 \\ 
    2 &   4 & B &   9 \\ 
    2 &   5 & B &   4 \\ 
    3 &   1 & A &  14 \\ 
    3 &   2 & A &   8 \\ 
    3 &   3 & A &  14 \\ 
    3 &   4 & A &  14 \\ 
    3 &   5 & A &  14 \\ 
    3 &   1 & B &   7 \\ 
    3 &   2 & B &   5 \\ 
    3 &   3 & B &   9 \\ 
    3 &   4 & B &   6 \\ 
    3 &   5 & B &   6 \\ 
   \hline
\end{tabular}
\quad
\begin{tabular}{cccc}
  \hline
group & trial & towel & weights \\ 
  \hline
    4 &   1 & A &  14 \\ 
    4 &   2 & A &  14 \\ 
    4 &   3 & A &  14 \\ 
    4 &   4 & A &  14 \\ 
    4 &   5 & A &  14 \\ 
    4 &   1 & B &  14 \\ 
    4 &   2 & B &  14 \\ 
    4 &   3 & B &  14 \\ 
    4 &   4 & B &  14 \\ 
    4 &   5 & B &  14 \\ 
    5 &   1 & A &  22 \\ 
    5 &   2 & A &  28 \\ 
    5 &   3 & A &  13 \\ 
    5 &   4 & A &  23 \\ 
    5 &   5 & A & NA \\ 
    5 &   1 & B &  14 \\ 
    5 &   2 & B &  11 \\ 
    5 &   3 & B &   4 \\ 
    5 &   4 & B &  11 \\ 
    5 &   5 & B &   3 \\ 
    6 &   1 & A &  14 \\ 
    6 &   2 & A &  14 \\ 
    6 &   3 & A &  14 \\ 
    6 &   4 & A &  14 \\ 
    6 &   5 & A &  14 \\ 
    6 &   1 & B &  10 \\ 
    6 &   2 & B &  11 \\ 
    6 &   3 & B &  13 \\ 
    6 &   4 & B &  12 \\ 
    6 &   5 & B &   7 \\ 
   \hline
\end{tabular}
\end{table}
 
\end{document}