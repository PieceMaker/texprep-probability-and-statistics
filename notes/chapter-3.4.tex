% Template from https://www.overleaf.com/latex/templates/chapter-review-notes/npqqbrvfkwqh
\documentclass[11pt]{article}
\usepackage[utf8]{inputenc}	
\usepackage{amsmath,amsthm,amsfonts,amssymb,amscd}
\usepackage{multirow,booktabs}
\usepackage[table]{xcolor}
\usepackage{fullpage}
\usepackage{lastpage}
\usepackage{enumitem}
\usepackage{fancyhdr}
\usepackage{mathrsfs}
\usepackage{wrapfig}
\usepackage{setspace}
\usepackage{calc}
\usepackage{multicol}
\usepackage{cancel}
\usepackage[margin=3cm]{geometry}
\usepackage{amsmath}
\newlength{\tabcont}
\setlength{\parindent}{0.0in}
\setlength{\parskip}{0.05in}
\usepackage{empheq}
\usepackage{framed}
\usepackage[most]{tcolorbox}
\usepackage{xcolor}
\colorlet{shadecolor}{orange!15}
\parindent 0in
\parskip 12pt
\geometry{margin=1in, headsep=0.25in}
\theoremstyle{definition}
\newtheorem{defn}{Definition}
\newtheorem{reg}{Rule}
\newtheorem{exer}{Exercise}
\newtheorem{note}{Note}
\begin{document}
\title{Title}

\thispagestyle{empty}

\begin{center}
{\LARGE \bf Chapter 3.4}\\
{\large Probability \& Statistics}
\end{center}

\section{Binomial Distribution}

Last week we learned about probability distributions and how they summarize the probabilities of all possible values of the random variable being studied. In most cases, it can be very expensive, if not impossible, to construct the probability distribution for a random variable.

One of the probability distributions we talked about was functional distributions, i.e. a function that defines the probability for each observable value. As long as it satisfies the three laws of probability, any function can define a probability distribution. Once a distribution has been defined, it can be analyzed and properties inferred. These properties can be used to determine what type of data the distribution is good at modeling.

The first functional distribution we will define is known as the Binomial Distribution. It is used to model \textit{binomial experiments}.

\begin{shaded}
	A \textit{binomial experiment} posses the following properties:
	\begin{enumerate}
		\item The experiment consists of a fixed number, $n$, of identical trials.
		\item Each trial results in one of two outcomes: success, $S$, or failure, $F$.
		\begin{enumerate}
			\item Flipping a coin.
			\item Each manufactured item is either defective or non-defective.
			\item Each shot at a shooting range is either a hit or a miss.
			\item Each insurance policy either has a loss or it does not.
		\end{enumerate}
		\item The probability of success on a single trial is equal to some value $p$ and remains the same from trial to trial. The probability of a failure is equal to $q = (1-p)$.
		\item The trials are independent, i.e. once an observation is made it does not change the likelihood of the results of the remaining experiments.
		\item The random variable of interest is $Y$, the number of successes observed during the $n$ trials.
	\end{enumerate}
\end{shaded}

A sample point from a binomial experiment measuring defective and non-defective engines (out of a total of 5 engines) at a factory might look like the following, where defective is a success $S$ and non-defective is a failure $F$ (success doesn't have to be a positive, just what we define it to be): $SFFSF$. In this case $Y=2$. It could look like $FFSFF$, with $Y=1$. It could also be $FFFFF$, with $Y=0$. We could also have all successes $SSSSS$, $Y=5$. Notice that the sample point $FFFSS$ also results in $Y=2$.

We now define the probability function for the Binomial Distribution as

$$
	p(y) = {n \choose y} p^y q^{n-y}, \; y = 0,1,2,\ldots,n \; \text{and} \; 0 \leq p \leq 1, \; q = 1-p
$$

First, we notice that every probability is greater than or equal to 0. This satisfies the first rule of probability. For the second rule, let's take a look back to the binomial expansion we looked at before the test. As a review, the binomial expansion is

$$
	(x + y)^n = {n \choose 0} x^n y^0 + {n \choose 1} x^{n-1}y^1 + {n \choose 2} x^{n-2}y^2 + \cdots + {n \choose n} x^0 y^n = \sum_{i=0}^n {n \choose i} x^{n-i}y^i
$$

Notice the relation with the Binomial Distribution function we defined and this binomial expansion. What if we set $x=q=1-p$ and $y=p$? Then we have

$$
	(q + p)^n = {n \choose 0} q^n p^0 + {n \choose 1} q^{n-1}p^1 + {n \choose 2} q^{n-2}p^2 + \cdots + {n \choose n} q^0 p^n = \sum_{i=0}^n {n \choose i} q^{n-i}p^i
$$

But since $q = 1-p$, then this equals

$$
	(q + p)^n = (1 - p + p)^n = 1^n = 1
$$

By this, we know that the binomial distribution function satisfies the second rule of probability.

\subsection{Electrical Fuses}

Suppose that a lot of 5000 electrical fuses contains 5\% defectives. If a sample of 5 fuses is tested, find the probability of observing at least one defective.

\subsubsection*{Solution}

It is reasonable to assume that $Y$, the number of defectives observed, has an approximate binomial distribution because the lot is large. Removing a few fuses does not change the composition of those remaining enough to cause us concern. Thus,

$$
	\begin{aligned}
		P(\text{at least one defective}) & = 1 - p(0) = 1 - {5 \choose 0} p^0q^5 \\
		& = 1 - (0.95)^5 = 1 - 0.774 = 0.226
	\end{aligned}
$$

Notice that there is a fairly large chance of seeing at least one defective, even though the sample is quite small.

\subsection{Electrical Fuses pt. 2}

Refer to the previous problem. What is the probability of observing at least one defective if the lot of fuses contains 25\% defectives?

\subsubsection*{Solution}

$$
	\begin{aligned}
		P(\text{at least one defective}) & = 1 - p(0) = 1 - {5 \choose 0} p^0q^5 \\
		& = 1 - (0.75)^5 = 1 - 0.2373047 = 0.7626953
	\end{aligned}
$$

\subsection{Identify Binomial Experiments}

In 2003, the average combined SAT score (math and verbal) for college-bound students in the US was 1026. Suppose that approximately 45\% of all high school graduates took this test and that 100 high school graduates are randomly selected from among all high school grads in the US. Which of the following random variables has a distribution that can be approximated by a binomial distribution?

\begin{enumerate}[label=\alph*)]
	\item The number of students who took the SAT
	\item The scores of the 100 students in the sample
	\item The number of students in the sample who scored above the average on the SAT
	\item The amount of time required by each student to complete the SAT
	\item The number of female high school grads in the sample
\end{enumerate}

Once a probability distribution has been defined, we have everything we need to calculate expected values. Therefore we can calculate $E[Y]$ and $V[Y]$. We will show how to calculate $E[Y]$.

\begin{shaded}
	$$
		\begin{aligned}
			E[Y] & = \sum_y yp(y) = \sum_{y=0}^n y {n \choose y} p^y q^{n-y} \\
			& = \sum_{n=1}^n y \frac{n!}{(n-y)!y!} p^y q^{n-y} \\
			& = \sum_{y=1}^n \frac{n!}{(n-y)!(y-1)!} p^y q^{n-y} \;\;\; \text{(Cancel y in y!)}
		\end{aligned}
	$$
	$$
		\begin{aligned}
			& = np \sum_{y=1}^n \frac{(n-1)!}{(n-y)!(y-1)!} p^{y-1}q^{n-y} \;\;\; \text{(Factor out np)} \\
			& = np \sum_{y=0}^{n-1} \frac{(n-1)!}{(n-1-z)!z!} p^z q^{n-1-z} \;\;\; \text{(Substitute z = y-1)} \\
			& = np \sum_{z=0}^{n-1} {n-1 \choose z} p^z q^{n-1-z}
		\end{aligned}
	$$
	\newpage
	Notice that this summation is the sum of a binomial distribution with $n-1$ trials. Because of this, it sums to 1 and we now have
	$$
		E[Y] = np
	$$
	Using similar methods, we can calculate
	$$
		V[Y] = npq = np(1-p)
	$$
\end{shaded}

\begin{note}
	Talk to the students about the interpretation of $E[Y]$.
\end{note}

\subsection{Electrical Fuses Expected Values}

For the electrical fuse example, calculate $E[Y]$ and $V[Y]$ if

\begin{enumerate}[label=\alph*)]
	\item 5\% of the lot are defective
	\item 25\% of the lot are defective
\end{enumerate}

\subsubsection*{Solution}

\begin{enumerate}[label=\alph*)]
	\item
		$$
			\begin{aligned}
				E[Y] & = np = 5 \cdot 0.05 = 0.25 \\
				V[Y] & = npq = np(1-p) = 5 \cdot 0.05 \cdot 0.95 = 0.2375
			\end{aligned}
		$$
	\item
		$$
			\begin{aligned}
				E[Y] & = np = 5 \cdot 0.25 = 1.25 \\
				V[Y] & = npq = np(1-p) = 5 \cdot 0.25 \cdot 0.75 = 0.9375
			\end{aligned}
		$$
\end{enumerate}



\end{document}