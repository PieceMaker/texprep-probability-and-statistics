% Template from https://www.overleaf.com/latex/templates/chapter-review-notes/npqqbrvfkwqh
\documentclass[11pt]{article}
\usepackage[utf8]{inputenc}	
\usepackage{amsmath,amsthm,amsfonts,amssymb,amscd}
\usepackage{multirow,booktabs}
\usepackage[table]{xcolor}
\usepackage{fullpage}
\usepackage{lastpage}
\usepackage{enumitem}
\usepackage{fancyhdr}
\usepackage{mathrsfs}
\usepackage{wrapfig}
\usepackage{setspace}
\usepackage{calc}
\usepackage{multicol}
\usepackage{cancel}
\usepackage[retainorgcmds]{IEEEtrantools}
\usepackage[margin=3cm]{geometry}
\usepackage{amsmath}
\newlength{\tabcont}
\setlength{\parindent}{0.0in}
\setlength{\parskip}{0.05in}
\usepackage{empheq}
\usepackage{framed}
\usepackage[most]{tcolorbox}
\usepackage{xcolor}
\colorlet{shadecolor}{orange!15}
\parindent 0in
\parskip 12pt
\geometry{margin=1in, headsep=0.25in}
\theoremstyle{definition}
\newtheorem{defn}{Definition}
\newtheorem{reg}{Rule}
\newtheorem{exer}{Exercise}
\newtheorem{note}{Note}
\begin{document}
\title{Title}

\thispagestyle{empty}

\begin{center}
{\LARGE \bf Chapter 2.1-2.4}\\
{\large Probability \& Statistics}
\end{center}
\section{Set Notation Review}
We rely heavily on sets when trying to calculate probabilities, so we will review some basic set theory.

\begin{note}
	For the following set notation review, add Venn diagrams and sample sets where useful.
\end{note}

Capital letters $A, B, C$ will denote sets of points while lower-case letters (likely with subscripts) will denote elements of sets.

$$
A = \{a_1, a_2, a_3\}
$$

$S$ will denote the set of all elements under consideration and will be known as the \textit{universal set}.

For two sets $A$ and $B$, if $A$ is contained in $B$, i.e. every point in $A$ is also in $B$, then we say $A$ is a subset of $B$, denoted $A \subset B$.

For two sets $A$ and $B$, we can create the union of the sets, denoted $A \cup B$, as the set containing the unique elements that occur in one or both of sets $A$ and $B$. In word problems, the union is usually indicated by the word \textit{or}.

For two sets $A$ and $B$, we can create the intersection of the sets, denoted $A \cap B$, as the set containing the unique elements that occur in both $A$ and $B$. In word problems, the intersection is usually indicated by the work \textit{and}.

\begin{note}
	Use the following sets for union and intersection example: $A = \{1, 2, 3, 3, 4, 6\},  B = \{5, 5, 6, 7, 8, 9, 10\}, S = \{1, 2, 3, 4, 5, 6, 7, 8, 9, 10\}$.
\end{note}

If $A$ is a subset of $S$, the universal set, then we define the complement of $A$, denoted $A^\complement$, as the set of all elements in $S$ that are not in $A$. A simple but powerful result is that $A \cup A^\complement = S$, i.e. the union of $A$ and its compliment is the universal set.

There are four other useful laws that we will use throughout this course. They are the \textit{distributive laws}:

$$
\begin{aligned}
	A \cap (B \cup C) & = (A \cap B) \cup (A \cap C) \\
	A \cup (B \cap C) & = (A \cup B) \cap (A \cup C)
\end{aligned}
$$

\noindent and \textit{DeMorgan's laws}:

$$
\begin{aligned}
	(A \cap B)^\complement & = A^\complement \cup B^\complement \\
	(A \cup B)^\complement & = A^\complement \cap B^\complement
\end{aligned}
$$

\section{Discrete Probability Model}

\begin{shaded}
	An \textit{experiment} is the process by which an observation is made. Each possible outcome is known as an \textit{event}. Events can contain multiple sub-events and are known as \textit{compound events}. The most basic of events--events that cannot be decomposed any further--are known as \textit{simple events} or \textit{sample points}.
\end{shaded}

Example throwing a fair die. The simple events or sample points are:

\begin{itemize}[label={}]
	\setlength\itemsep{0em}
	\item $E_1$: Observe a 1 (simple event/sample point)
	\item $E_2$: Observe a 2
	\item $E_3$: Observe a 3
	\item $E_4$: Observe a 4
	\item $E_5$: Observe a 5
	\item $E_6$: Observe a 6
	\item $A$: Observe an odd number (compound event)
	\item $B$: Observe an even number
	\item $C$: Observe a number less than 5
	\item $D$: Observe a 2 or a 3
\end{itemize}

The \textit{sample space} ($S$) is made up of all the possible sample points. In the previous example, $S = \{E_1, E_2, E_3, E_4, E_5, E_6\}$.

\begin{note}
	Draw Venn diagram showing sample space and circled events.
\end{note}

A more complex example would be an experiment measuring the number of bacteria growing in a Petri dish.

\begin{itemize}[label={}]
	\setlength\itemsep{0em}
	\item $E_0$: Observe 0 bacteria
	\item $E_{64}$: Observe 64 bacteria
	\item $E_i$: Observe $i$ bacteria, $i \geq 0$
	\item $A$: Observe more than 200 bacteria
	\item $B$: Observe between 100 and 300 bacteria
\end{itemize}

When an experiment is conducted a single time, you will observe one and only one simple event. For example, with the bacteria experiment, if we observe 100 bacteria ($E_{100}$), then we cannot observe 99 bacteria ($E_{99}$) or 101 bacteria ($E_{101}$). Therefore, simple events or sample points are mutually exclusive events.

This finally brings us to the three axioms of probability.

\begin{shaded}
	Let $S$ be a sample space associated with an experiment. To every event $A$ in $S$, $A \subset S$, we assign a number, $P(A)$, called the \textit{probability} of $A$, so that the following three axioms hold:
	\begin{itemize}[label={}]
		\setlength\itemsep{0em}
		\item Axiom 1: $P(A) \geq 0$
		\item Axiom 2: $P(S) = 1$
		\item Axiom 3: If $A_1, A_2, \ldots$ form a sequence of pairwise mutually exclusive events in $S$, i.e. $A_i \cap A_j = \emptyset, i \neq j$, then
		$$
			P(A_1 \cup A_2 \cup A_3 \cup \ldots) = \sum_{i=1}^\infty P(A_i)
		$$
	\end{itemize}
\end{shaded}

\subsection{Light Bulb Manufacturing Example}

A manufacturer has five seemingly identical light bulbs available for shipping. Unknown to her, two of the five are defective. A particular order calls for two of the light bulbs is filled by randomly selecting two of the five that are available.

\begin{enumerate}
	\item List the sample space for this experiment.
	\item Let $A$ denote the event that the order is filled with two non-defective light bulbs. List the sample points in $A$.
	\item Construct a Venn diagram for the experiment that illustrates event $A$.
	\item Assign probabilities to the simple events in such a way that the information about the experiment is used and the axioms are met.
	\item Find the probability of event $A$.
\end{enumerate}

\subsubsection*{Solution}

\begin{enumerate}
	\item Let the two defective light bulbs be labeled $D_1$ and $D_2$ and let the three good light bulbs be labeled $G_1$, $G_2$, and $G_3$. Any single sample point will consist of a list of the two light bulbs selected for shipment. The simple events may be denoted by \\
	\begin{tabular}{cccc}
		$E_1 = \{D_1, D_2\}$ & $E_5 = \{D_2, G_1\}$ & $E_8 = \{G_1, G_2\}$ & $E_{10} = \{G_2, G_3\}$  \\
		$E_2 = \{D_1, G_1\}$ & $E_6 = \{D_2, G_2\}$ & $E_9 = \{G_1, G_3\}$ &  \\
		$E_3 = \{D_1, G_2\}$ & $E_7 = \{D_2, G_3\}$ &  &  \\
		$E_4 = \{D_1, G_3\}$ &  &  &  \\
	\end{tabular} \\
	Thus, there are ten sample points in $S$, and $S = \{E_1, E_2, \ldots, E_{10}\}$.
	\item Event $A = \{E_8, E_9, E_{10}\}$.
	\item Draw diagram on board.
	\item Because the light bulbs are selected at random, any pair of light bulbs is as likely to be selected as any other pair. Thus, $P(E_i) = \frac{1}{10}$, for $i=1, 2, \ldots, 10$, is a reasonable assignment of probabilities.
	\item Because $A = E_8 \cup E_9 \cup E_{10}$, Axiom 3 implies that
	$$
		P(A) = P(E_8) + P(E_9) + P(E_{10}) = \frac{1}{10} + \frac{1}{10} + \frac{1}{10} = \frac{3}{10}
	$$
\end{enumerate}

\end{document}