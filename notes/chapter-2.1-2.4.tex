% Template from https://www.overleaf.com/latex/templates/chapter-review-notes/npqqbrvfkwqh
\documentclass[11pt]{article}
\usepackage[utf8]{inputenc}	
\usepackage{amsmath,amsthm,amsfonts,amssymb,amscd}
\usepackage{multirow,booktabs}
\usepackage[table]{xcolor}
\usepackage{fullpage}
\usepackage{lastpage}
\usepackage{enumitem}
\usepackage{fancyhdr}
\usepackage{mathrsfs}
\usepackage{wrapfig}
\usepackage{setspace}
\usepackage{calc}
\usepackage{multicol}
\usepackage{cancel}
\usepackage[retainorgcmds]{IEEEtrantools}
\usepackage[margin=3cm]{geometry}
\usepackage{amsmath}
\newlength{\tabcont}
\setlength{\parindent}{0.0in}
\setlength{\parskip}{0.05in}
\usepackage{empheq}
\usepackage{framed}
\usepackage[most]{tcolorbox}
\usepackage{xcolor}
\colorlet{shadecolor}{orange!15}
\parindent 0in
\parskip 12pt
\geometry{margin=1in, headsep=0.25in}
\theoremstyle{definition}
\newtheorem{defn}{Definition}
\newtheorem{reg}{Rule}
\newtheorem{exer}{Exercise}
\newtheorem{note}{Note}
\begin{document}
\setcounter{section}{8}
\title{Title}

\thispagestyle{empty}

\begin{center}
{\LARGE \bf Chapter 2.1-2.4}\\
{\large Probability \& Statistics}
\end{center}
\section{Set Notation Review}
We rely heavily on sets when trying to calculate probabilities, so we will review some basic set theory.

\begin{note}
	For the following set notation review, add Venn diagrams and sample sets where useful.
\end{note}

Capital letters $A, B, C$ will denote sets of points while lower-case letters (likely with subscripts) will denote elements of sets.

$$
A = \{a_1, a_2, a_3\}
$$

$S$ will denote the set of all elements under consideration and will be known as the \textit{universal set}.

For two sets $A$ and $B$, if $A$ is contained in $B$, i.e. every point in $A$ is also in $B$, then we say $A$ is a subset of $B$, denoted $A \subset B$.

For two sets $A$ and $B$, we can create the union of the sets, denoted $A \cup B$, as the set containing the unique elements that occur in one or both of sets $A$ and $B$.

\begin{note}
	Use the following sets for union example: $A = \{1, 2, 3, 3, 4, 6\},  B = \{5, 5, 6, 7, 8, 9, 10\}$.
\end{note}

\end{document}