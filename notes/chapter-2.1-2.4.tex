% Template from https://www.overleaf.com/latex/templates/chapter-review-notes/npqqbrvfkwqh
\documentclass[11pt]{article}
\usepackage{float}
\usepackage[utf8]{inputenc}	
\usepackage{amsmath,amsthm,amsfonts,amssymb,amscd}
\usepackage{multirow,booktabs}
\usepackage[table]{xcolor}
\usepackage{fullpage}
\usepackage{lastpage}
\usepackage{enumitem}
\usepackage{fancyhdr}
\usepackage{mathrsfs}
\usepackage{wrapfig}
\usepackage{setspace}
\usepackage{calc}
\usepackage{multicol}
\usepackage{cancel}
\usepackage[margin=3cm]{geometry}
\usepackage{amsmath}
\newlength{\tabcont}
\setlength{\parindent}{0.0in}
\setlength{\parskip}{0.05in}
\usepackage{empheq}
\usepackage{framed}
\usepackage[most]{tcolorbox}
\usepackage{xcolor}
\colorlet{shadecolor}{orange!15}
\parindent 0in
\parskip 12pt
\geometry{margin=1in, headsep=0.25in}
\theoremstyle{definition}
\newtheorem{defn}{Definition}
\newtheorem{reg}{Rule}
\newtheorem{exer}{Exercise}
\newtheorem{note}{Note}
\begin{document}
\title{Title}

\thispagestyle{empty}

\begin{center}
{\LARGE \bf Chapter 2.1-2.4}\\
{\large Probability \& Statistics}
\end{center}
\section{Set Notation Review}
We rely heavily on sets when trying to calculate probabilities, so we will review some basic set theory.

\begin{note}
	For the following set notation review, add Venn diagrams and sample sets where useful.
\end{note}

Capital letters $A, B, C$ will denote sets of points while lower-case letters (likely with subscripts) will denote elements of sets.

$$
A = \{a_1, a_2, a_3\}
$$

$S$ will denote the set of all elements under consideration and will be known as the \textit{universal set}.

For two sets $A$ and $B$, if $A$ is contained in $B$, i.e. every point in $A$ is also in $B$, then we say $A$ is a subset of $B$, denoted $A \subset B$.

For two sets $A$ and $B$, we can create the union of the sets, denoted $A \cup B$, as the set containing the unique elements that occur in one or both of sets $A$ and $B$. In word problems, the union is usually indicated by the word \textit{or}.

For two sets $A$ and $B$, we can create the intersection of the sets, denoted $A \cap B$, as the set containing the unique elements that occur in both $A$ and $B$. In word problems, the intersection is usually indicated by the work \textit{and}.

\begin{note}
	Use the following sets for union and intersection example: $A = \{1, 2, 3, 3, 4, 6\},  B = \{5, 5, 6, 7, 8, 9, 10\}, S = \{1, 2, 3, 4, 5, 6, 7, 8, 9, 10\}$.
\end{note}

If $A$ is a subset of $S$, the universal set, then we define the complement of $A$, denoted $A^\complement$, as the set of all elements in $S$ that are not in $A$. A simple but powerful result is that $A \cup A^\complement = S$, i.e. the union of $A$ and its compliment is the universal set.

There are four other useful laws that we will use throughout this course. They are the \textit{distributive laws}:

$$
\begin{aligned}
	A \cap (B \cup C) & = (A \cap B) \cup (A \cap C) \\
	A \cup (B \cap C) & = (A \cup B) \cap (A \cup C)
\end{aligned}
$$

\noindent and \textit{DeMorgan's laws}:

$$
\begin{aligned}
	(A \cap B)^\complement & = A^\complement \cup B^\complement \\
	(A \cup B)^\complement & = A^\complement \cap B^\complement
\end{aligned}
$$

\section{Discrete Probability Model}

\begin{shaded}
	An \textit{experiment} is the process by which an observation is made. Each possible outcome is known as an \textit{event}. Events can contain multiple sub-events and are known as \textit{compound events}. The most basic of events--events that cannot be decomposed any further--are known as \textit{simple events} or \textit{sample points}.
\end{shaded}

Example throwing a fair die. The simple events or sample points are:

\begin{itemize}[label={}]
	\setlength\itemsep{0em}
	\item $E_1$: Observe a 1 (simple event/sample point)
	\item $E_2$: Observe a 2
	\item $E_3$: Observe a 3
	\item $E_4$: Observe a 4
	\item $E_5$: Observe a 5
	\item $E_6$: Observe a 6
	\item $A$: Observe an odd number (compound event)
	\item $B$: Observe an even number
	\item $C$: Observe a number less than 5
	\item $D$: Observe a 2 or a 3
\end{itemize}

The \textit{sample space} ($S$) is made up of all the possible sample points. In the previous example, $S = \{E_1, E_2, E_3, E_4, E_5, E_6\}$.

\begin{note}
	Draw Venn diagram showing sample space and circled events.
\end{note}

A more complex example would be an experiment measuring the number of bacteria growing in a Petri dish.

\begin{itemize}[label={}]
	\setlength\itemsep{0em}
	\item $E_0$: Observe 0 bacteria
	\item $E_{64}$: Observe 64 bacteria
	\item $E_i$: Observe $i$ bacteria, $i \geq 0$
	\item $A$: Observe more than 200 bacteria
	\item $B$: Observe between 100 and 300 bacteria
\end{itemize}

When an experiment is conducted a single time, you will observe one and only one simple event. For example, with the bacteria experiment, if we observe 100 bacteria ($E_{100}$), then we cannot observe 99 bacteria ($E_{99}$) or 101 bacteria ($E_{101}$). Therefore, simple events or sample points are mutually exclusive events.

This finally brings us to the three axioms of probability.

\begin{shaded}
	Let $S$ be a sample space associated with an experiment. To every event $A$ in $S$, $A \subset S$, we assign a number, $P(A)$, called the \textit{probability} of $A$, so that the following three axioms hold:
	\begin{itemize}[label={}]
		\setlength\itemsep{0em}
		\item Axiom 1: $P(A) \geq 0$
		\item Axiom 2: $P(S) = 1$
		\item Axiom 3: If $A_1, A_2, \ldots$ form a sequence of pairwise mutually exclusive events in $S$, i.e. $A_i \cap A_j = \emptyset, i \neq j$, then
		$$
			P(A_1 \cup A_2 \cup A_3 \cup \ldots) = \sum_{i=1}^\infty P(A_i)
		$$
	\end{itemize}
\end{shaded}

\subsection{Applying Axioms}

A sample space consists of five simple events, $E_1$, $E_2$, $E_3$, $E_4$ and $E_5$.

\begin{enumerate}
	\item If $P(E_1) = P(E_2) = 0.15$, $P(E_3) = 0.4$, and $P(E_4) = 2P(E_5)$, find the probabilities of $E_4$ and $E_5$.
	\item If $P(E_1) = 3P(E_2) = 0.3$, find the probabilities of the remaining simple events if you know that the remaining simple events are equally probable.
\end{enumerate}

\subsubsection*{Solution}

\begin{enumerate}
	\item By axioms 2 and 3,
		$$
			\begin{aligned}
				P(S) & = P(E_1) + P(E_2) + P(E_3) + P(E_4) + P(E_5) \\
				& = 0.15 + 0.15 + 0.4 + P(E_4) + P(E_5) \\
				& = 0.7 + 2P(E_5) + P(E_5) \\
				& = 0.7 + 3P(E_5) = 1 \\
				\therefore P(E_5) & = \frac{0.3}{3} = 0.1 \\
				\therefore P(E_4) & = 2P(E_5) = 2\cdot0.1 = 0.2
			\end{aligned}
		$$
	\item "The remaining simple events are equally probable" means $P(E_3)=P(E_4)=P(E_5)$. Applying this with axioms 2 and 3,
		$$
			\begin{aligned}
				P(S) & = P(E_1) + P(E_2) + P(E_3) + P(E_4) + P(E_5) \\
				& = 0.3 + \frac{0.3}{3} + 3P(E_3) \\
				& = 0.4 + 3P(E_3) = 1 \\
				\therefore P(E_3) & = \frac{0.6}{3} = 0.2 = P(E_4) = P(E_5)
			\end{aligned}
		$$
\end{enumerate}

\subsection{Light Bulb Manufacturing Example}

A vehicle manufacturer has five seemingly identical engines available for shipping. Unknown to her, two of the five are defective. A mechanic calls and orders two engines. The order is filled by randomly selecting two of the five engines that are available.

\begin{enumerate}
	\item List the sample space for this experiment.
	\item Let $A$ denote the event that the order is filled with two non-defective engines. List the sample points in $A$.
	\item Assign probabilities to the simple events in such a way that the information about the experiment is used and the three axioms/rules are met.
	\item Find the probability of event $A$.
\end{enumerate}

\subsubsection*{Solution}

\begin{enumerate}
	\item Let the two defective engines be labeled $D_1$ and $D_2$ and let the three good engines be labeled $G_1$, $G_2$, and $G_3$. Any single sample point will consist of a list of the two engines selected for shipment. The simple events may be denoted by \\
	\begin{tabular}{cccc}
		$E_1 = \{D_1, D_2\}$ & $E_5 = \{D_2, G_1\}$ & $E_8 = \{G_1, G_2\}$ & $E_{10} = \{G_2, G_3\}$  \\
		$E_2 = \{D_1, G_1\}$ & $E_6 = \{D_2, G_2\}$ & $E_9 = \{G_1, G_3\}$ &  \\
		$E_3 = \{D_1, G_2\}$ & $E_7 = \{D_2, G_3\}$ &  &  \\
		$E_4 = \{D_1, G_3\}$ &  &  &  \\
	\end{tabular} \\
	Thus, there are ten sample points in $S$, and $S = \{E_1, E_2, \ldots, E_{10}\}$.
	\begin{note}
		Talk to the students about why these are the simple events and not each engine. For example, why is $E_1 \cap E_2 = \emptyset$ instead of $\{D_1\}$.
	\end{note}
	\item Event $A = \{E_8, E_9, E_{10}\}$.
	\item Because the engines are selected at random, any pair of engines is as likely to be selected as any other pair. Thus, $P(E_i) = \frac{1}{10}$, for $i=1, 2, \ldots, 10$, is a reasonable assignment of probabilities. This choice of probabilities satisfies axioms 1 and 2.
	\item Because $A = E_8 \cup E_9 \cup E_{10}$, and each of the simple events is mutually exclusive, Axiom 3 implies that
	$$
		P(A) = P(E_8) + P(E_9) + P(E_{10}) = \frac{1}{10} + \frac{1}{10} + \frac{1}{10} = \frac{3}{10}
	$$
\end{enumerate}

\subsection{Eye Glass Example}

A survey classified a large number of adults according to whether they were diagnosed as needing eyeglasses to correct their reading vision and whether they use eyeglasses when reading. The proportions falling into the four resulting categories are given in the following table:

\begin{table}[H]
	\centering
	\begin{tabular}{lcc}
		\hline
		& \multicolumn{2}{c}{Uses Eyeglasses} \\
		& \multicolumn{2}{c}{for Reading} \\\cmidrule{2-3}
		Needs glasses & Yes & No \\\hline
		Yes & 0.44 & 0.14 \\
		No & 0.02 & 0.40 \\\hline
	\end{tabular}
\end{table}

\noindent If a single adult is selected from the large group, find the probabilities of the following events. The adult

\begin{enumerate}
	\item $A$: needs glasses.
	\item $B$: needs glasses but does not use them.
	\item $C$: uses glasses whether the glasses are needed or not.
\end{enumerate}

\subsubsection*{Solution}

The simple events are as follows:

\begin{itemize}
	\item $E_1$: The person needs glasses and uses them for reading. $P(E_1) = 0.44$
	\item $E_2$: The person needs glasses and does not use them for reading. $P(E_2) = 0.14$
	\item $E_3$: The person does not need glasses and uses them for reading. $P(E_3) = 0.02$
	\item $E_4$: The person does not need glasses and does not use them for  reading. $P(E_4) = 0.40$
\end{itemize}

\begin{enumerate}
	\item $A = \{E_1, E_2\} \therefore P(A) = P(E_1) + P(E_2) = 0.44 + 0.14 = 0.58$
	\item $B = E_2 \therefore P(B) = P(E_2) = 0.14$
	\item $C = \{E_1, E_3\} \therefore P(C) = P(E_1) + P(E_3) = 0.44 + 0.02 = 0.46$
\end{enumerate}

\subsection{Coin Toss}

A balanced coin is tossed four times. Calculate the probability that at least one toss results in heads.

\subsubsection*{Solution}

\begin{enumerate}
	\item To start we need to define the sample space.\\
	\begin{table}[H]
	\centering
		\begin{tabular}{ccccc}
			$E_1 = \{HHHH\}$ & $E_2 = \{HHHT\}$ & $E_6 = \{HHTT\}$ & &  \\
			& $E_3 = \{HHTH\}$ & $E_7 = \{HTHT\}$ & &  \\
			& $E_4 = \{HTHH\}$ & $E_8 = \{HTTH\}$ & $E_{12} = \{HTTT\}$ &  \\
			& $E_5 = \{THHH\}$ & $E_9 = \{THHT\}$ & $E_{13} = \{THTT\}$ &  \\
			& & $E_{10} = \{THTH\}$ & $E_{14} = \{TTHT\}$ &  \\
			& & $E_{11} = \{TTHH\}$ & $E_{15} = \{TTTH\}$ & $E_{16} = \{TTTT\}$  \\
		\end{tabular} \\
	\end{table}
	\item Now that we have our sample space, we need to assign probabilities to each event. Since the coin is balanced, we can assume each simple event is equally likely, so $P(E_i) = \frac{1}{16}$.
	\item It's time to calculate our probability. The event of interest $A$ is
	$$
		A = \{E_1, E_2, \ldots, E_{15}\}
	$$
	This is a lot of probabilities to add up. But $A^\complement = \{E_{16}\}$ and $A \cup A^\complement = S$. Therefore $1 = P(S) = P(A) + P(A^\complement)$ and $P(A) = 1 - P(A^\complement) = 1 - \frac{1}{16} = \frac{15}{16}$.
\end{enumerate}

\end{document}