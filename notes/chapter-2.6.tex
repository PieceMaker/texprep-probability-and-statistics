% Template from https://www.overleaf.com/latex/templates/chapter-review-notes/npqqbrvfkwqh
\documentclass[11pt]{article}
\usepackage{float}
\usepackage[utf8]{inputenc}	
\usepackage{amsmath,amsthm,amsfonts,amssymb,amscd}
\usepackage{multirow,booktabs}
\usepackage[table]{xcolor}
\usepackage{fullpage}
\usepackage{lastpage}
\usepackage{enumitem}
\usepackage{fancyhdr}
\usepackage{mathrsfs}
\usepackage{wrapfig}
\usepackage{setspace}
\usepackage{calc}
\usepackage{multicol}
\usepackage{cancel}
\usepackage[margin=3cm]{geometry}
\usepackage{amsmath}
\newlength{\tabcont}
\setlength{\parindent}{0.0in}
\setlength{\parskip}{0.05in}
\usepackage{empheq}
\usepackage{framed}
\usepackage[most]{tcolorbox}
\usepackage{xcolor}
\colorlet{shadecolor}{orange!15}
\parindent 0in
\parskip 12pt
\geometry{margin=1in, headsep=0.25in}
\theoremstyle{definition}
\newtheorem{defn}{Definition}
\newtheorem{reg}{Rule}
\newtheorem{exer}{Exercise}
\newtheorem{note}{Note}
\begin{document}
\title{Title}

\thispagestyle{empty}

\begin{center}
{\LARGE \bf Chapter 2.6}\\
{\large Probability \& Statistics}
\end{center}
\section{Counting Techniques}

Last week we talked about sample spaces and how, if you can construct the sample space, then you can determine probabilities of events. However, the coin flipping example showed just how tedious it can be to construct a sample space very single time. Instead, we can use counting techniques to determine the size of a sample space and the number of simple events satisfying any event we wish to calculate the probability of.

Our goal with these techniques is to assume that every simple event in our sample space is equiprobable. If this is the case, then we only need to calculate two numbers for any probability. $N$, the size of the sample space, i.e. the number of simple events in the sample space and, for event of interest $A$, we need to calculate $n_a$, the number of simple events in $A$. If we can calculate these numbers, then we have the probability of our event as $P(A) = \frac{n_a}{N}$.

The first rule will be called the $mn rule$.

\begin{shaded}
	With $m$ elements $a_1, a_2, \ldots, a_m$ and $n$ elements $b_1, b_2, \ldots, b_n$, it is possible to form $mn = m \cdot n$ pairs containing one element from each group.
\end{shaded}

\begin{note}
	Show table with $m$ columns and $n$ rows and show how the formula makes sense visually.
\end{note}

The prior rule can be extended to any number of sets. For example, if there were a third set containing elements $c_1, c_2, \ldots, c_p$, then there would be $m \cdot n \cdot p$ triples containing one element from each group.

\subsection{Dice Example}

An experiment involves tossing a pair of dice and observing the numbers on the upper faces. Find the number of sample points in $S$, the sample space for the experiment.

\subsubsection*{Solution}

A sample point in this experiment is an ordered pair of numbers representing the value of each die in the roll. We can therefore apply the $mn rule$ to calculate the size of $S$.

For this experiment, $m = n = 6$ because there are six sides (and values) on each die. Therefore the size of $S$ is $N = mn = 6\cdot6 = 36$.

\subsection{Coin Flip Pt. 2}

Think back to the coin flipping problem from last week where we flipped a coin 4 times and observed the result. We calculated the simple events with order mattering and calculated the sample space. There were 16 simple events in the sample space. Do we get the same answer using the $mn rule$?

\subsubsection*{Solution}

Each coin flip has 2 possibilities, a head or a tail. Each flip can be viewed as a separate group. Therefore, we can apply the $mn$ rule with 4 groups and we get

$$
	N = 2 \cdot 2 \cdot 2 \cdot 2 = 2^4 = 16
$$

\subsection{Birthday Problem}

Including the teacher and assistant, there are 20 people in this class. Discounting leap years, what is the probability of at least two people having the same birthday?

\subsubsection*{Solution}

We are interested in the event $A$, at least two people have the same birthday. This encompasses 2 people, 3 people, etc. all the way up to all 20 people sharing the same birthday. What is an easier way to calculate this probability? What is $A^\complement$?

Since we are ignoring leap years, there are $1, 2, \ldots, 365$ days to choose from. We will view an observation in this experiment as an ordered sequence of 20 numbers where the first number denotes the birthday of the first person, the second number the birthday of the second person, etc.

In order to calculate the probability, we must first calculate the size of the sample space. Since each experiment views 20 people (groups) and each person has 365 possibilities, then we can apply the $mn$ rule and we get

$$
	N = 365^{20}
$$

Now, we need to find the size of $A^\complement$. For this, let us first find the number of days to choose from for each person. We can pick any of the 365 day for the first person. What about for the second person? If we already chose a date and the second person cannot have the same birthday for the event to be in $A^\complement$, there are 364 days to choose from. There are 363 for the third. Continuing on, there are 346 to choose from for the 20th person. Each of these can be viewed as the size of the group for each person. Therefore we can apply the $mn$ rule and get

$$
	n_a = 365 \cdot 364 \cdot 363 \cdot \ldots \cdot 346
$$

Therefore,

$$
	\begin{aligned}
		P\left(A^\complement\right) & = \frac{n_a}{N} = \frac{365 \cdot 364 \cdot 363 \cdot \ldots \cdot 346}{365^{20}} = 0.5886 \\
		\therefore P(A) & = 1 - P\left(A^\complement\right) = 0.4114
	\end{aligned}
$$

\indent There is a special name for the technique we used to calculate $n_a$ in the previous problem. Before we introduce it, let us first introduce the concept of \textit{factorial}.

\begin{shaded}
	We define factorial of an integer $n$, denoted $n!$, as
	$$
		n! = n \cdot (n-1) \cdot (n-2) \cdot \ldots \cdot 2 \cdot 1
	$$
	For convenience, we define
	$$
		0! = 1
	$$
\end{shaded}

Now that we have the concept of factorial, we introduce \textit{permutation}.

\begin{shaded}
	An ordered arrangement of $r$ distinct objects is called a \textit{permutation}. The number of ways of ordering $n$ distinct objects taken $r$ at a time will be designated by the symbol $P^n_r$.
\end{shaded}

Based on the birthday problem, we can calculate the number of permutations as

$$
	P^n_r = n(n-1)(n-2)\cdots(n-r+1) = \frac{n!}{(n-r)!}
$$

\subsection{Employee Door Prizes}

A company has 30 employees. A company decides to have a random drawing for 3 prizes. The first name drawn gets \$100, the second name \$50, and the third name \$25. How many sample points are associated with this experiment?

\subsubsection*{Solution}

Because the prizes awarded are different, the number of sample points is the number of ordered arrangements of $r=3$ out of the possible $n=30$ names. Thus, the number of sample points in $S$ is

$$
	P^{30}_3 = \frac{30!}{(30-3)!} = \frac{30!}{27!} = 30 \cdot 29 \cdot 28 = 24360
$$

\subsection{Manufacturer Efficiency}

Suppose a manufacturer identifies 4 tasks in the manufacturing process that can be performed in any order. The manufacturer knows that changing the order each of these tasks is performed can change the efficiency of the manufacturing process. How many orderings of these tasks would the manufacturer have to check to find the most efficient process?

\subsubsection*{Solution}

Since the order each task is performed can change the efficiency of the process, we have to check all ordered pairs. The number of such pairs is

$$
	P^4_4 = \frac{4!}{(4-4)!} = \frac{4!}{0!} = 4! = 24
$$

\begin{shaded}
	The number of \textit{combinations} of $n$ objects taken $r$ at a time is the number of subsets, each of size $r$, that can be formed from the $n$ objects. This number will be denoted by $n \choose r$.
\end{shaded}

The number of combinations is calculated as

$$
	{n \choose r} = \frac{n!}{r! (n-r)!}
$$

\subsection{Job Applicants}

Suppose a company is hiring for 4 identical positions and they get 20 applications. How many different ways can these positions be filled?

\subsubsection*{Solution}

In this case, order does not matter because we do not consider hiring person 1  then person 2 to be different from hiring person 2 then person 1. Therefore the number of ways of filling these positions is

$$
	{20 \choose 4} = \frac{20!}{4!(20-4)!} = \frac{20!}{4!16!} = \frac{20 \cdot 19 \cdot 18 \cdot 17}{4 \cdot 3 \cdot 2 \cdot 1} = 19 \cdot 17 \cdot 5 \cdot  3 = 4845
$$

\subsection{Engine Problem (ctd.)}

A vehicle manufacturer has five seemingly identical engines available for shipping. Unknown to her, two of the five are defective. A mechanic calls and orders two engines. The order is filled by randomly selecting two of the five engines that are available. What is the probability that the order is filled with two non-defective engines?

\subsubsection*{Solution}

The last time we worked this problem we found there were 10 simple events in the sample space and 3 of them consist of a pair of non-defective engines. Since the order of the engines in the pair does not matter, we can use combinations to calculate this.

For the size of the sample space $N$, there are 5 engines to choose from and we are selecting sets of size 2. Therefore,

$$
	N = {5 \choose 2} = \frac{5!}{2!(5-2)!} = \frac{5!}{2!3!} = \frac{5*4}{2!} = \frac{20}{2} = 10
$$

Now, to calculate the size of the event of interest, $n_a$, we realize that we are still selecting sets of size 2, but now we only have 3 engines to choose from because there are only 3 non-defective engines. This gives us

$$
	n_a = {3 \choose 2} = \frac{3!}{2!(3-2!)} = \frac{3!}{2!1!} = 3
$$

Using our definition of probability, we have

$$
	P(A) = \frac{n_a}{N} = \frac{3}{10} = 0.3
$$

This matches what we calculated last time.

\subsection{Traveling Salesman Problem}

Suppose a salesman wishes to fly to 10 cities in 10 days. The distances between each city is known. The price of each flight changes on each day and the price from city A to B is different than the price from city B to A. The salesman knows all prices. How many distances are there? If the salesman wishes to minimize the cost of the trip, how many routes must be checked?

\subsubsection*{Solution}

The distances do not change and they are the same when considering the distance from city A to B as the distance from B to A. Therefore we are interested in all unordered pairs. This is a combination problem.

$$
	{10\choose2} = \frac{10!}{2!8!} = \frac{10 \cdot 9}{2} = 45
$$

Next, since the flight prices change daily and the price from city A to B differs from city B to A, order matters when calculating the price of each journey. Therefore, we have to consider all ordered sets of the 10 cities. The total number of these is

$$
	P^{10}_{10} = \frac{10!}{(10-10)!} = 10! = 3,628,800
$$

Counting combinations can be generalized to multiple groups of different sizes, as long as those sizes add to the total number of objects.

\begin{shaded}
	The number of ways of partitions $n$ distinct objects into $k$ distinct groups containing $n_1, n_2, \ldots, n_k$ objects respectively, where each object appears in exactly one group and $\sum_{i=1}^k n_i = n$ is
	
	$$
		N = {n \choose n_1 n_2 \cdots n_k } = \frac{n!}{n_1!n_2!\cdots n_k!}
	$$
\end{shaded}

Taking a step back, there is an application of the combination formula used in expanding a polynomial of the form $(x + y)^n$. The following equality holds:

$$
	(x + y)^n = {n \choose 0} x^n y^0 + {n \choose 1} x^{n-1}y^1 + {n \choose 2} x^{n-2}y^2 + \cdots + {n \choose n} x^0 y^n = \sum_{i=0}^n {n \choose i} x^{n-i}y^i
$$

\subsection{Polynomial Expansion}

Expand the polynomial $(x + y)^3$ by hand and using the previous formula.

\subsubsection*{Solution}

\begin{enumerate}
	\item
		$$
			\begin{aligned}
				(x + y)^3 & = (x^2 + 2xy + y^2)(x + y) \\
				& = x^3 + 2x^2y + xy^2 + x^2y + 2xy^2 + y^3 \\
				& = x^3 + 3x^2y + 3xy^2 + y^3
			\end{aligned}
		$$
	\item
		$$
			\begin{aligned}
				(x + y)^3 & = \sum_{i=0}^3 {3 \choose i} x^{3-i}y^i \\
				& = {3 \choose 0} x^3y^0 + {3 \choose 1} x^2y^1 + {3 \choose 2} x^1y^2 + {3 \choose 3} x^0y^3 \\
				& = x^3 + 3x^2y + 3xy^2 + y^3
			\end{aligned}
		$$
\end{enumerate}

This is also used to create Pascal's Triangle.

\begin{table}[H]
	\centering
	\begin{tabular}{>{$n=}l<{$\hspace{12pt}}*{13}{c}}	
		0 &&&&&&&1&&&&&&\\
		1 &&&&&&1&&1&&&&&\\
		2 &&&&&1&&2&&1&&&&\\
		3 &&&&1&&3&&3&&1&&&\\
		4 &&&1&&4&&6&&4&&1&&\\
		5 &&1&&5&&10&&10&&5&&1&\\
		6 &1&&6&&15&&20&&15&&6&&1
	\end{tabular}
\end{table}

Each row makes up all the possible numbers of combinations for the given $n$. Notice that each column in a row is made up by summing the adjacent columns on the previous row. This is a quick way of determining the numbers of combinations.

\subsection{Employment Dispute}

A labor dispute has arisen concerning the assignment of 20 workers to four different construction jobs. The first job (considered to be very undesirable) required 6 workers. The second, third, and fourth required 4, 5, and 4 workers respectively. The dispute arose over an alleged random assignment of the laborers to the jobs that placed all 4 members of a particular ethnic group on job 1, the undesirable job. In considering whether the assignment represented injustice, a mediation panel desired the probability of the observed event.

Determine the number of sample points in the sample space for this experiment, i.e. how many ways 20 people can be assigned to the 4 jobs. Finally, find the probability of the observed event if it is assumed the workers are randomly assigned.

\subsubsection*{Solution}

The number of ways of assigning the 20 workers to the 4 jobs is equal to the number of ways of partitioning 20 workers into 4 groups of sizes $n_1 = 6, n_2 = 4, n_ 3 = n_4 = 5$. This is

$$
	N = {20 \choose 6 4 5 5} = \frac{20!}{6!4!5!5!}
$$

Now, the number of sample points resulting in all 4 members of the ethnic group being assigned to job 1 starts out with placing all members on job 1. We then need to count the number of ways the remaining 16 employees can be assigned to the 4 jobs. There are 16 employees to choose from and the remaining spots on job 1 is 2, since we already assigned 4 people to job 1. Therefore

$$
	n_a = {16 \choose 2 4 5 5} = \frac{16!}{2!4!5!5!}
$$

We then have

$$
	P(A) = \frac{n_a}{N} = \frac{\frac{16!}{2!4!5!5!}}{\frac{20!}{6!4!5!5!}} = \frac{16!6!}{20!2!} \approx 0.0031
$$

\subsection{Dice Tossing Example}

An experiment consists of tossing a pair of dice.

\begin{enumerate}
	\item Use the counting techniques to determine the number of sample points in the sample space $S$.
	\item Find the probability that the sum of the numbers appearing on the dice is equal to 7.
\end{enumerate}

\subsubsection*{Solution}

\begin{enumerate}
	\item In this problem we want every sample point to be equally probable. If order didn't matter, then we would only calculate one sample point for ${1, 2}$. However, there are 2 ways this can occur while there is only 1 way that ${1, 1}$ can occur. Therefore, we cannot assign equal probabilities to these and we must assume order matters. Therefore, we apply the $mn$ rule and get
	$$
		N = 6 \cdot 6 = 36	
	$$
	\item The numbers in an observed pair are dependent on each other in order to satisfy the requirement that the sum of the dice is 7. For example, if the first die is a 4, then the second die \textit{must} be a 3. This is not an application of permutations or combinations and a direct application of the $mn$ rule is not readily apparent. Therefore we have to get creative.
	
	7 is an odd number so it will always be made of separate numbers on each die. Therefore we do not count snake eyes etc. which would only be counted once. Therefore, for each pair of numbers we see we have to add 2 to $n_a$ since the flipped order pair must also be counted. The pairs satisfying the event of interest are {1,6}, {2,5}, and {3,4}. There are 3 of these, but we have to double them. Therefore there are 6 possible ways of rolling two die summing to 7 and we have the probability of the event as
	
	$$
		P(A) = \frac{n_a}{N} = \frac{6}{36} = \frac{1}{6}
	$$
\end{enumerate}

\subsection{Telephone Numbers}

How many different seven-digit telephone numbers can be formed if the first digit cannot be 0?

\subsubsection*{Solution}

Each digit can be viewed as a separate group. The first group has 9 to choose from, the remaining groups have 10 to choose from. Application of the expanded $mn$ rule gives us

$$
	N = 9 \cdot 10^6
$$

\subsection{Raffle Tickets}

A local sorority is conducting a raffle where 50 tickets are to be sold--one per customer. There are three prizes to be awarded. If there are four organizers of the raffle and they each buy one ticket, what is the probability of the following events:

\begin{enumerate}
	\item The organizers win all of the prizes.
	\item The organizers win exactly 2 of the prizes.
	\item The organizers win exactly 1 of the prizes.
	\item The organizers win none of the prizes.
\end{enumerate}

\subsubsection*{Solution}

Our sample space does not change so we use the same $N$ in calculating all 4 of these probabilities. Notice that the probabilities do not care who wins which prize, just that specific people win \textit{a} prize. Therefore order does not matter and we calculate $N$ using combinations:

$$
	N = {50 \choose 3} = \frac{50!}{3!47!} = \frac{50\cdot49\cdot48}{3\cdot2\cdot1} = 19600
$$

\begin{enumerate}
	\item Call this event $A$. If the organizers win all the prizes, then there are 4 people who can win and we are interested in sets of 3 of them.
	$$
		\begin{aligned}
			n_a = {4 \choose 3} = \frac{4!}{3!1!} = 4 \\
			\therefore P(A) = \frac{4}{19600} = 0.0002040816
		\end{aligned}
	$$
	\item Call this event $B$. If the organizers win exactly 2 prizes, then we want all sets that contain 2 organizers and 1 of the 46 non-organizers. In this case,
	$$
		\begin{aligned}
			n_b = {4 \choose 2} {46 \choose 1} = \frac{4!}{2!2!} \cdot \frac{46!}{1!45!} = 6 \cdot 46 = 276 \\
			\therefore P(B) =  \frac{276}{19600} = 0.01408163
		\end{aligned}
	$$
	Note, we multiply these two sets of combinations due to the $mn$ rule.
	\item Call this event $C$. Similar to part (2), we calculate $n_c$ as
	$$
		\begin{aligned}
			n_c = {4 \choose 1} {46 \choose 2} = \frac{4!}{1!3!} \cdot \frac{46!}{2!44!} = 4 \cdot 1035 = 4140 \\
			\therefore P(C) =  \frac{4140}{19600} = 0.2112245
		\end{aligned}
	$$
	\item Call this event $D$. We only consider non-organizers, so
	$$
		\begin{aligned}
			n_d = {46 \choose 3} = \frac{46!}{3!45!} = \frac{46 \cdot 45 \cdot 44}{3 \cdot 2 \cdot 1} = 15180 \\
			\therefore P(C) =  \frac{15180}{19600} = 0.7744898
		\end{aligned}
	$$
\end{enumerate}

\subsection{All 6 Faces of A Die}

A balanced die is tossed six times, and the number on the uppermost face is recorded each time. What is the probability that the numbers recorded are 1, 2, 3, 4, 5, and 6 in any order?

\subsubsection*{Solution}

Each throw of the die can be viewed as a separate group with 6 possibilities. Therefore, using the extended $mn$-rule, we have

$$
	N = 6^6 = 46656
$$

To find $n_a$, we realize that order matters since we used the $mn$-rule to calculate $N$. Therefore, we have

$$
	n_a = P^6_6 = 6! = 720
$$

The probability of the event of interest is

$$
	P(A) = \frac{n_a}{N} = \frac{720}{46656} = 0.0154321
$$

\subsection{Straights in Poker}

Five cards are dealt from a standard 52-card deck. What is the probability that we draw

\begin{enumerate}
	\item 1 ace, 1 two, 1 three, 1 four, and 1 five?
	\item any straight?
\end{enumerate}

\subsubsection*{Solution}

In both examples, we have

$$
	N = {52 \choose 5} = \frac{52!}{5!47!} = \frac{52\cdot51\cdot50\cdot49\cdot48}{5\cdot4\cdot3\cdot2\cdot1}
$$

\begin{enumerate}
	\item We calculated $N$ using unordered sets. There are 4 ways we can get an ace, 4 ways a two, 4 ways a three, 4 ways a four, and 4 ways a five. We do not care about order so
	$$
		\begin{aligned}
			n_a = 4^5 = 1024 \\
			\therefore P(A) = \frac{1024}{2598960} = 0.0003940038
		\end{aligned}
	$$
	\item Straights can begin with (starting from the smallest card in the straight) ace, two, three, four, five, six, seven, eight, nine, and ten. Therefore, the probability of any straight is
	$$
		P(\text{any straight}) = 10 \cdot P(A) = 0.003940038
	$$
\end{enumerate}

\subsection{Student Government Positions}

A group of three undergraduate and five graduate students are available to fill certain student government posts. If four students are to be randomly selected from this group, find the probability

\begin{enumerate}
	\item that exactly two undergraduates will be among the four chosen.
	\item that at least two undergraduates will be among the four chosen.
\end{enumerate}

\subsubsection*{Solution}

$$
	N = {8 \choose 4} = \frac{8!}{4!4!} = \frac{8\cdot7\cdot6\cdot5}{4\cdot3\cdot2\cdot1} = 70
$$

\begin{enumerate}
	\item To satisfy the event of interest we need to make sure two (and only two) undergraduates are chosen. There are three to choose from and we want to select two. The remaining will be graduates. This will then be
	$$
		\begin{aligned}
			n_3 = {3 \choose 2}{5 \choose 2} = 3 \cdot \frac{5!}{2!3!} = 3 \cdot \frac{5 \cdot 4}{2} = 30 \\
			\therefore P(\text{exactly 2 undergraduates}) = \frac{30}{70} = \frac{3}{7}
		\end{aligned}
	$$
	\item We need to calculate the probability of exactly 3 undergraduates. The probability of exactly 4 undergraduates is 0 because there are not 4 undergraduates to choose from.
	$$
		\begin{aligned}
			n_4 & = {3 \choose 3}{5 \choose 1} = 1 \cdot 5 = 5 \\
			\therefore P(\text{exactly 2 undergraduates}) & = \frac{5}{70} \\
			\therefore P(\geq \text{2 undergraduates}) & = P(\text{exactly 2 undergraduates}) + P(\text{exactly 3 undergraduates}) \\
			& = \frac{30}{70} + \frac{5}{70} = \frac{35}{70} = \text{0.5 or 50\%}
		\end{aligned}
	$$
\end{enumerate}

\end{document}