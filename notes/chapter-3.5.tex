% Template from https://www.overleaf.com/latex/templates/chapter-review-notes/npqqbrvfkwqh
\documentclass[11pt]{article}
\usepackage[utf8]{inputenc}	
\usepackage{amsmath,amsthm,amsfonts,amssymb,amscd}
\usepackage{multirow,booktabs}
\usepackage[table]{xcolor}
\usepackage{fullpage}
\usepackage{lastpage}
\usepackage{enumitem}
\usepackage{fancyhdr}
\usepackage{mathrsfs}
\usepackage{wrapfig}
\usepackage{setspace}
\usepackage{calc}
\usepackage{multicol}
\usepackage{cancel}
\usepackage[margin=3cm]{geometry}
\usepackage{amsmath}
\newlength{\tabcont}
\setlength{\parindent}{0.0in}
\setlength{\parskip}{0.05in}
\usepackage{empheq}
\usepackage{framed}
\usepackage[most]{tcolorbox}
\usepackage{xcolor}
\colorlet{shadecolor}{orange!15}
\parindent 0in
\parskip 12pt
\geometry{margin=1in, headsep=0.25in}
\theoremstyle{definition}
\newtheorem{defn}{Definition}
\newtheorem{reg}{Rule}
\newtheorem{exer}{Exercise}
\newtheorem{note}{Note}
\begin{document}
\title{Title}

\thispagestyle{empty}

\begin{center}
{\LARGE \bf Chapter 3.5}\\
{\large Probability \& Statistics}
\end{center}

\section{Geometric Distribution}

The last distribution we will talk about is the geometric distribution. To motivate this distribution, let us review the properties of binomial experiments and modify them slightly. For this new type of experiment, we will continue to have identical and independent trials, each of which resulting in either a success or a failure. The probability of success will also continue to remain the same in each trial. The key difference for this new type of experiment is that, instead of having a fixed number of trials that will be run and observed, we run trials until we get the first success. The random variable $Y$ will denote the number of the trial that resulted in the first success. This random variable $Y$ is called a Geometric random variable.

Consider the following examples of possible sample points:

$$
	\begin{aligned}
		E_1: & S & \text{(success on first trial) } & (Y=1) \\
		E_2: & FS & \text{(success on second trial) } & (Y=2) \\
		E_3: & FFS & \text{(success on third trial) } & (Y=3) \\
		E_{10}: & FFFFFFFFFS & \text{(success on tenth trial) } & (Y=10) \\
		\cdots & & &
	\end{aligned}
$$

\begin{shaded}
	The probability distribution representing this type of variable is the \textit{geometric distribution} and is denoted as
	$$
		p(y) = q^{y-1}p, \;\;\; y=1, 2, 3, \ldots \;\;\; 0 \leq p \leq 1
	$$
\end{shaded}

Clearly this distribution satisfies the first probability rule. What about the second?

$$
	\begin{aligned}
		\sum_{y=1}^\infty p(y) & = \sum_{y=1}^\infty q^{y-1}p = p \sum_{y=1}^\infty q^{y-1} \\
		& = p \sum_{z=0}^\infty q^z = \frac{p}{1-q} = \frac{p}{1-(1-p)} = 1
	\end{aligned}
$$

This last equality holds because we know the following sum:

$$
	\sum_{z=0}^\infty r^z = \frac{1}{1-r}, \;\;\; |r| < 1
$$

\end{document}